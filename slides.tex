\documentclass[10pt]{beamer}
\usetheme[navigation]{UMONS}
\usepackage[utf8]{inputenc}
\usepackage[english]{babel}
\usepackage[no-math]{fontspec}%
\usepackage{xeCJK}
%\setCJKmainfont{SimSun}
\setCJKsansfont[BoldFont={黑体}, ItalicFont={楷体}]{方正兰亭黑简体}
\usepackage{lipsum}
\usepackage{makecell}
\usepackage{amsmath}
\usepackage{mathtools}
\usepackage{amsfonts,amssymb}
\usepackage{subfigure}
\usepackage{enumerate}
\usepackage{graphicx}
\usepackage{booktabs}
\usepackage{textpos}
\newcommand{\slideheight}{6.7cm}
\newcommand{\slidewidth}{12cm}
\newcommand{\incicon}[1]{\includegraphics[height=0.7cm]{figures/#1}
%\renewcommand\CJKfamilydefault{\CJKrmdefault}
	
}

\title[China University of Petroleum(East China)]{Classification Method of Lake Water in Remote Sensing Image Based on Support Vector Machine}
\subtitle{基于支持向量机的遥感影像湖泊水体分类方法}
\author[D.Wang]{答辩人:王迪}
\institute[Geomatics Engineering]{%
 China University of Petroleum(East China)\\
 中国石油大学(华东)\\[1ex]
 School of Geosciences
 \\地球科学与技术学院\\[1ex]
 Geomatics Engineering Class 1402\\
 测绘工程1402班
  \\[2ex]
  \includegraphics[height=12ex]{./figure/UPC.pdf}
  \includegraphics[height=12ex]{./figure/dy.pdf}
  \includegraphics[height=12ex]{./figure/1402.pdf}
}
\date{June 19, 2018}
\begin{document}
\maketitle
\section{简介}
\subsection{简介}

\begin{frame}
\frametitle{简介}
\framesubtitle{研究背景}
%	\small
%	\item [\uppercase\expandafter{\romannumeral1}] 水资源是一种重要的经济资源和战略资源,全球水资源丰富,然而淡水资源却十分稀缺,在小到区域,大到全球的范围内,如何对水资源实行合理高效监控一直是人们关心的热点问题。
%	
%	\item [\uppercase\expandafter{\romannumeral2}]通过发射遥感资源卫星实行对地观测是一种日渐成熟的方法,其中的代表有美国的陆地卫星计划(Landsat)等,它具有覆盖范围广,周期性强等特点,适合进行大区域长时间序列的宏观变化研究。
%	
%	\item [\uppercase\expandafter{\romannumeral3}]通过对地观测获取的遥感影像能够准确地表示水体所蕴含的光谱信息和纹理、边界、形状及拓扑关系等几何特性,在水体监测中具有明显的优势。
%	
%	\item [\uppercase\expandafter{\romannumeral4}]利用遥感影像进行水体范围提取,对其进行量化分析,进而研究该区域水资源变化状况,已经得到了越来越多人的重视,从而探究如何进行高精度水体提取成为当前亟待解决的重点问题。
%	
%	\item [\uppercase\expandafter{\romannumeral5}]湖北省是水资源非常丰富的一个内陆省,省内湖泊密布,有“千湖之省”的称号,本文以湖北省内的三个大型湖泊为研究对象,借助Landsat卫星获取的遥感影像,进行湖泊水体提取和分类的研究。
为了进行湖泊区域水体变化检测,衡量人类活动的影响
\begin{figure}
	\small
	\includegraphics[width=0.5\textwidth]{./figure/beijing.pdf}
	\caption{Lake and Surrounding Dense Fishponds(R, G, B-NIR, R, G)}
\end{figure}
\end{frame}

%\begin{frame}
%	\frametitle{简介}
%	\framesubtitle{研究现状}
%	\begin{figure}
%		\includegraphics[height=5.5cm]{./figure/yanjiuxianzhuang.pdf}
%	\end{figure}
%\end{frame}
%
%\begin{frame}
%	\frametitle{简介}
%	\framesubtitle{研究现状}
%	\begin{block}{水体提取指标}
%	\begin{itemize}
%	\item 通过Landsat-TM传感器的绿色波段和近红外波段组合形成归一化差异水体指标(Normalized
%	Difference Water Index, NDWI)来勾绘水体特征(McFeeters, 1996)。
%	\item 提出改进的归一化差异水体指标(Modified Normalized DifferenceWater Index, MNDWI), 该指标的波段组合表达式和NDWI相同,不同的是中红外波段代替了近红外波段(Xu, 2006)。
%	\item Shen(2010)通过水体目标的如下规律“TM传感器绿色波段与红色波段之和大于TM近红外波段与中红外波段之和”给出了含水率指标(Water Ratio Index, WRI)。
%	\item 提出一种自动化水体提取指标(Automated Water Extraction Index, AWEI),该指标通过最大化水与非水像素在不同波段之间的分离度来形成(Feyisa, 2014)。
%	\end{itemize}
%	\end{block}
%\end{frame}
\begin{frame}
\frametitle{简介}
\framesubtitle{研究现状}
\begin{block}{机器学习方法}
	\begin{itemize}
	\small
	\item 通过水体指标提取结合主成分分析(Principal Component Analysis, PCA)对伊朗乌尔米耶湖进行变化检测(Rokni, 2014)。
	\item \alert{Tran(2015)对越南陈文泰县湄公河三角洲的水产养殖场采用最大似然分类法(Maximun Likelihood Classification, MLC)进行提取并检测区域变化}
	\item 利用传统的支持向量机(Support Vector Machine, SVM)分类方法,通过光谱特性对鄱阳湖湿地进行了长期的变化分析(Han, 2015)。
	\item 通过构建专家系统(Expert System)研究了近32年全世界水资源变化状况(Pekel, 2016)。
	\item 定义规则的随机森林(Random Forest, RF)模型,采集水体进行提取指标变换后的特征,分析了武汉市周边湖泊多年的变化(Deng, 2017)。
	\item \alert{对中国及越南的四个河口三角洲使用连通成分分割(Connected Component Segmentation) 的方法分离出鱼塘(Ottinger, 2017)。}
	\end{itemize}
\end{block}
\end{frame}

\begin{frame}
\frametitle{简介}
\framesubtitle{研究内容}
\begin{block}{目的}
区分湖泊自然水体与人工水体(如鱼塘)
\end{block}
\begin{block}{问题}
光谱相近
\end{block}
\begin{block}{思想}
\begin{itemize}
	\item 面向对象:对象\raisebox{0.1mm}{-----}同类地物像素的聚集
	\item 光谱特征 \raisebox{0.1mm}{+} 几何特征
\end{itemize}
\end{block}
\end{frame}


\section{基本信息}
\subsection{基本信息}

%\begin{frame}
%	\frametitle{模型}
%	\begin{block}{C-SVC}
%	\[
%	\begin{aligned}
%	\min\limits_{\omega,b,\xi}\quad & \frac{1}{2}\omega^T\omega+C\sum_{i=1}^{l}\xi_i\\
%	\text{s.t.}\quad & y^{(i)}(\omega^Tx^{(i)}+b) \geq 1-\xi_i,\quad i=1,\cdots,l\\
%	& \xi_i \geq 0,\quad i=1,\cdots,l
%	\end{aligned}
%	\]
%	$\omega,b$是\alert{分割超平面}的参数
%	\end{block}
%	\begin{block}{高斯核函数}
%	\[K(x,y)=\exp(\frac{-||x-y||^2}{2\sigma^2})\]
%	\end{block}
%	
%\end{frame}

\begin{frame}[fragile]
\frametitle{数据}
数据从\alert{USGS官网}获取。\\
\begin{table}
\small
\newcommand{\tabincell}[2]{\begin{tabular}{@{}#1@{}}#2\end{tabular}} 
\centering
\caption{Data Table}
\begin{tabular}{cccccc}
		\toprule
		Satellite 	& Sensor & row/col & Date 	& Resolution(m)&Adoption\\ \midrule
		Landsat-1&  MSS&132/039 &1973.12.08&60&Sensor Consistency\\ 
		Landsat-3&	MSS&133/039 &1983.02.08&60&Water Extraction\\
		Landsat-5&	TM&	123/039	&1995.12.05&30&Sensor Consistency\\
		Landsat-5&	TM&	123/039	&1998.10.26&30&Water Extraction\\
		Landsat-5&	TM&	123/039	&2011.01.15&30&SVM(testing)\\
		Landsat-7& ETM\raisebox{0.4mm}{+}&123/039 &1999.12.24&30&Water Extraction\\
		Landsat-7& ETM\raisebox{0.4mm}{+}&123/039 &2001.01.11&30&Sensor Consistency\\
		Landsat-8&	OLI&123/039 &2013.12.22&30&Sensor Consistency\\
		Landsat-8&	OLI&123/039 &2014.01.23&30&SVM(training)\\
		Landsat-8&	OLI&122/039 &2015.01.03&30&SVM(testing)\\
		Landsat-8&	OLI & 123/039 & 2015.03.31 & 30 & \tabincell{c}{SVM(testing)\\Water Extraction}\\ \bottomrule
	\end{tabular}
\end{table}
\end{frame}

\begin{frame}
 \frametitle{研究区}
研究区位于中国湖北省,选择洪湖、斧头湖、梁子湖。
 \begin{figure}
 \centering
  \includegraphics[height=5.5cm]{./figure/yanjiuquyu.pdf}
  \caption{Honghu Lake, Futouhu Lake and Liangzihu Lake}
 \end{figure}
\end{frame}

\begin{frame}
\frametitle{准备工作}
\begin{block}{训练集}
	\begin{itemize}
	\item 创建:在ENVI软件中对影像选取感兴趣区域(Region of Interest, ROI)。\\
	\item 结果:总共276个对象,12个水体对象以及264个鱼塘对象。
	\end{itemize}
\end{block}
\begin{block}{LIBSVM}
	\begin{itemize}
		\item 由台湾大学Chih-Jen Lin教授开发的SVM程序库。
		\item 训练、建模、数据输入和预测。
		\item 使用在MATLAB中的接口。
		\item 了解更多:\url{https://www.csie.ntu.edu.tw/~cjlin/libsvm/}
	\end{itemize}
\end{block}
\end{frame}

\section{方法}
\subsection{方法}

%\begin{frame}
% \frametitle{预处理}
% \centering
% \includegraphics[height=\slideheight]{./figure/yuchuliliucheng_en.pdf}
%\end{frame}

\begin{frame}
 \frametitle{水体提取 \& 分类}
 \centering
 \includegraphics[height=\slideheight]{./figure/tiqufenleiliucheng_en.pdf}
\end{frame}

\newcommand{\masterpicheight}{5cm}

%\begin{frame}[fragile]
% \frametitle{水体提取}
% \framesubtitle{人工确定指标}
%\begin{table}
%	\small
%	\newcommand{\tabincell}[2]{\begin{tabular}{@{}#1@{}}#2\end{tabular}}  
%	\centering
%	\begin{tabular}{ccc}
%		\toprule
%		Index\footnote{K.Rokni, et al. Water Feature Extraction and Change Detection
%			Using Multitemporal Landsat Imagery. Remote Sensing, 2014}	& Formula & Signal\\ \midrule
%		NDWI&\tabincell{c}{(G-NIR)/(G+NIR)}&Water>Threshold\\ 
%		NDMI&\tabincell{c}{(NIR-MIR)/(NIR+MIR)}&Water>Threshold\\
%		MNDWI&\tabincell{c}{(G-MIR)/(G+MIR)}&Water>Threshold\\
%		WRI&\tabincell{c}{(G+R)/(NIR+MIR)}&Water>1\\
%		NDVI&\tabincell{c}{(NIR-R)/(NIR+R)}&Water<Threshold\\
%		\tabincell{c}{AWEI}&\tabincell{c}{4$\times$(G-MIR)-\\(0.25$\times$ NIR+2.75$\times$SWIR)}&\tabincell{c}{Water>Threshold}
%		\\ \bottomrule
%	\end{tabular}
%	\label{index_inf}
%\end{table}
%\begin{block}{试错法}
%	\begin{itemize}
%	\item 初步确定阈值
%	\[Threshold=\arg \min |S_{Extraction}-S_{Reference}|\]
%	\end{itemize}
%\end{block}
%\end{frame}

%\begin{frame}
% \frametitle{水体提取}
% \framesubtitle{人工确定指标}
% \begin{figure}
%  	\centering
% 	\includegraphics[height=3cm]{./figure/tiquyuzhi.pdf}
% 	\caption{Threshold Certained  for TM, ETM\raisebox{0.2mm}{+}, and OLI Sensor Manually}
% \end{figure}
% \begin{block}{精度评定}
% 	\begin{itemize}
% 	\item Shapefile$\Rightarrow$Raster
% 	\item 混淆矩阵
% 	\end{itemize}
% \end{block}
%\end{frame}

\begin{frame}
 \frametitle{水体提取}
 \framesubtitle{自动确定阈值}
\begin{columns}
	\begin{column}{0.4\textwidth}
	\footnotesize
	\alert{直方图区域分割法}\\
	假设$t$将像元值分为$k_1,k_2$两类($\leq t,>t$)。$k_i$类有$n_i$个像元,$\mu_i,\sigma_i$分别是$k_i$的均值和方差, $m$是整个图像灰度的平均值。
	\[m=\frac{\mu_1n_1+\mu_2n_2}{n_1+n_2}\]
	类内方差:
	\[\sigma_W^2=n_1\sigma_1^2+n_2\sigma_2^2\]
	类间方差:
	\[\sigma_B^2=n_1(\mu_1-m)^2+n_2(\mu_2-m)^2\]
	阈值:
	\[T=\arg \max\frac{\sigma_B^2}{\sigma_W^2}\]
	\end{column}
	\begin{column}{0.6\textwidth}
	\begin{figure}
	\flushleft
	\includegraphics[height=5.7cm]{./figure/histogram.pdf}
	\end{figure}
	\end{column}
\end{columns}
\end{frame}

\begin{frame}
	 \frametitle{水体提取}
 \framesubtitle{结果}
 \begin{figure}
 	\centering
 	\subfigure[原始影像]{
 		\includegraphics[width=0.3\textwidth,angle=180]{./figure/lzh_OLI.pdf}
 	}
 	\subfigure[分割结果]{
 		\includegraphics[width=0.3\textwidth,angle=180]{./figure/lzh_split.pdf}
 	}
 	\subfigure[提取结果]{
 		\includegraphics[width=0.3\textwidth,angle=180]{./figure/lzh_extract.pdf}
 	}
 \end{figure}
\end{frame}

%\begin{frame}
%\frametitle{特征获取}
%\framesubtitle{对象扫描及更新}
%\begin{figure}
%	\centering
%	\includegraphics[width=0.9\textwidth]{./figure/duixiangsaomiao.pdf}
%\end{figure}
%\end{frame}
%
%\begin{frame}
%\frametitle{特征获取}
%\framesubtitle{计算特征}
%\begin{block}{光谱信息}
%	\small
%	\begin{itemize}
%	\item 光谱一致性分析
%	\begin{itemize}
%	\small
%	\item[-] MSS/TM/ETM-OLI
%	\item[-] 12月-1月
%	\item[-] 泥、沙、水、植被
%	\item[-] 5*5窗口的平均反射率(MSS取3*3)
%	\end{itemize}
%		\begin{figure}
%		\footnotesize
%		\setcounter{subfigure}{0}
%		\centering
%		\subfigure[MSS-OLI]{
%			\includegraphics[width=0.2\textwidth,angle=-90]{./figure/mss_oli.pdf}
%		}
%		\subfigure[TM-OLI]{
%			\includegraphics[width=0.2\textwidth,angle=-90]{./figure/tm_oli.pdf}
%		}
%		\subfigure[ETM-OLI]{
%			\includegraphics[width=0.2\textwidth,angle=-90]{./figure/etm_oli.pdf}
%		}
%	\end{figure}
%	\end{itemize}
%	\end{block}
%\end{frame}
%\begin{frame}
%\frametitle{特征获取}
%\framesubtitle{计算特征}
%\begin{block}{光谱信息}
%		\begin{itemize}
%		\item 光谱一致性分析\footnote{X.Han, et al. Four decades of winter wetland changes in Poyang Lake based on Landsat observations between 1973 and 2013. Remote Sensing of Environment, 2015.}
%		\begin{itemize}
%		\item[*] 计算\alert{相关系数}
%		\item[*] 多项式拟合:
%		\[y=\theta_0 x+\theta_1\]
%		\item[*] 光谱数据转换: MSS/TM/ETM$\Rightarrow$OLI
%		\end{itemize}
%		\item 光谱特征
%		\begin{itemize}
%		\item[-] 均值(E)
%		\item[-] 方差(Var)
%		\end{itemize}
%%		\end{itemize}
%\end{block}
%\end{frame}

\begin{frame}
\frametitle{特征获取}
\framesubtitle{计算特征}
\small
光谱特征:各波段均值(E),方差(Var)
\begin{block}{几何特征}
	\footnotesize
	\begin{itemize}
		\item 面积(A), 周长(P)
		\item 紧凑系数(Compactness C), P2A\footnote{R.S.Montero, et al. State of the art of compactness and circularity measures. Psychoanalysis \& Contemporary Science, 2009.}
		\[Compactness \quad C =\sqrt{\frac{4\pi A}{P^2}}\]
		
		 \[P2A=\frac{P^2}{A}\]
		\item 矩形度(Rectangularity)
		\[Rectangularity=\frac{A}{A_{rec}}\]
		$A_{rec}$是对象\alert{最小外接矩形}的面积。
	\end{itemize}
\end{block}
\end{frame}

\begin{frame}
\frametitle{特征获取}
\framesubtitle{计算特征}
\begin{block}{几何特征}
	\begin{itemize}
		\item 对象弯曲度(Curvature)
		\[K=\frac{|y^{''}|}{(1+y^{'2})^{\frac{3}{2}}},\quad Curvature=\frac{\text{number of }(K=0)}{\text{number of }(\text{Edge Pixels})}\]
				\begin{figure}
			\centering
			\includegraphics[width=0.28\textwidth]{./figure/frame.pdf}
			\includegraphics[width=0.37\textwidth]{./figure/cur.pdf}
		\end{figure}
	\end{itemize}
\end{block}
\end{frame}

\begin{frame}
\frametitle{分类}
\framesubtitle{数据预处理}
\begin{block}{特征归一化}
	\begin{itemize}
	\item 线性归一化:将数据归一化到$[l,u]$
	\[x_{ij}^{norm}=l+(u-l)\frac{x_j^{(i)}-\min x_{:j}}{\max x_{:j}-\min x_{:j}}\]
	$x_{:j}$是包含所有对象第$j$个特征的向量。
	\end{itemize}
\end{block}
\begin{block}{确定惩罚系数$C$及$\gamma$}
	\begin{itemize}
		\item LIBSVM-grid.py
			\[\gamma=\frac{1}{2\sigma^2}\]
	\end{itemize}
\end{block}
\end{frame}

\begin{frame}
	\frametitle{分类}
	\framesubtitle{特征组合-英文简称说明}
	\begin{columns}
		\begin{column}{0.4\textwidth}
		\begin{block}{Spec-光谱特征}
		\begin{itemize}
			\small
			\item E-对象各波段均值
			\\[2ex]
			\item Var-对象各波段方差
		\end{itemize}
		\end{block}
		\end{column}
		\begin{column}{0.5\textwidth}
		\begin{block}{Geo-几何特征}
		\begin{itemize}
			\small
			\item Basic-基本几何特征
			\\[2ex]
			Advanced-进阶几何特征
			\\[2ex]
			\item Macro-宏观角度
			\\[2ex]
			Micro-微观角度
			\\[2ex]
			\item P-对象周长,A-对象面积
			\\[2ex]
			P2A \& Compactness C-对象紧凑性
			\\[2ex]
			rec-矩形度,cur-对象弯曲度
		\end{itemize}
		\end{block}
		\end{column}
	\end{columns}
\end{frame}

\begin{frame}
	\frametitle{分类}
	\framesubtitle{特征组合}
	\begin{table}[htbp]
		\small
		\centering
		\begin{tabular}{ccccccccc}
			\toprule
			& \multicolumn{2}{c}{Spec}& \multicolumn{6}{c}{Geo}\\ 
			\cmidrule(lr){2-3} \cmidrule(lr){4-9}
			&   &   & \multicolumn{2}{c}{Basic}&\multicolumn{4}{c}{Advanced}\\ 
			\cmidrule(lr){4-5} \cmidrule(lr){6-9}
			&   &   &   &   & \multicolumn{3}{c}{Macro}&Micro\\
			\cmidrule(lr){6-8} \cmidrule(lr){9-9}
			& E &Var	&P &A &P2A &Compactness C &rec &cur \\ \midrule
			Spec&$\surd$&$\surd$&   &   &   &    & &\\
			P+A&	&		& $\surd$ & $\surd$   &   &  &    &\\
			Spec+P+A&$\surd$&$\surd$& $\surd$ & $\surd$   &   &  &    &\\
			P+A+ger\footnote{O.Macro, et al. Large-Scale Assessment of Coastal Aquaculture
				Ponds with Sentinel-1 Time Series Data. Remote Sensing, 2017}&	&		& $\surd$ & $\surd$   &$\surd$&$\surd$&    &\\
			P+A+rec&	&		& $\surd$ & $\surd$   &   &  &$\surd$     &\\
			P+A+cur&	&		& $\surd$ & $\surd$   &   &  &    &$\surd$ \\ \bottomrule
		\end{tabular}
		\label{feature_combine}
	\end{table}
\end{frame}

\section{结果}
\subsection{结果}

\begin{frame}
	\frametitle{结果}
	\framesubtitle{水体提取}
	原始影像:R, G, B-NIR, R, G
	\begin{figure}
		\setcounter{subfigure}{0}
		\small
		\centering
		\subfigure[原始影像]{
			\begin{minipage}[t]{0.2\textwidth}
				\includegraphics[width=1\textwidth]{./figure/lzh_origin_dist.pdf}\\
				\includegraphics[width=1\textwidth]{./figure/fth_origin_dist.pdf}
			\end{minipage}
		}
		\subfigure[提取结果]{
			\begin{minipage}[t]{0.2\textwidth}
				\includegraphics[width=1\textwidth]{./figure/lzh_extract_dist.pdf}\\
				\includegraphics[width=1\textwidth]{./figure/fth_extract_dist.pdf}
			\end{minipage}
		}
	\end{figure}
\end{frame}

\begin{frame}
	\frametitle{结果}
	\framesubtitle{分类(P+A+cur)}
	原始影像:R, G, B-NIR, R, G
	
	分类结果:蓝色-自然水体,青色-鱼塘
	\begin{figure}
		\setcounter{subfigure}{0}
		\small
		\centering
		\subfigure[区域1]{
			\includegraphics[width=0.14\textwidth]{./figure/classify_result/1_origin.pdf}
			\includegraphics[width=0.14\textwidth]{./figure/classify_result/1_classify.pdf}
		}
		\subfigure[区域2]{
			\includegraphics[width=0.15\textwidth]{./figure/classify_result/2_origin.pdf}
			\includegraphics[width=0.15\textwidth]{./figure/classify_result/2_classify.pdf}
		}
		\subfigure[区域3]{
			\includegraphics[width=0.15\textwidth]{./figure/classify_result/3_origin.pdf}
			\includegraphics[width=0.15\textwidth]{./figure/classify_result/3_classify.pdf}
		}
		\subfigure[区域4]{
			\includegraphics[width=0.15\textwidth]{./figure/classify_result/4_origin.pdf}
			\includegraphics[width=0.15\textwidth]{./figure/classify_result/4_classify.pdf}
		}
		\subfigure[区域5]{
			\includegraphics[width=0.15\textwidth]{./figure/classify_result/5_origin.pdf}
			\includegraphics[width=0.15\textwidth]{./figure/classify_result/5_classify.pdf}
		}
	\end{figure}
\end{frame}

\begin{frame}
	\frametitle{结果}
	\framesubtitle{分类(P+A+cur)}
	原始影像:R, G, B-NIR, R, G
	
	分类结果:蓝色-自然水体,青色-鱼塘
	\begin{figure}
		\centering
		\includegraphics[width=0.4\textwidth]{./figure/lzh_dong_GF_0122_origin.pdf}
		\includegraphics[width=0.4\textwidth]{./figure/lzh_dong_GF_0122_classify.pdf}
	\end{figure}
\end{frame}

\section{精度分析}
\subsection{精度分析}
%\begin{frame}[fragile]
%	\frametitle{精度评价}
%	\framesubtitle{水体提取-确定指标}
%	处理了Landsat MSS, TM, ETM\raisebox{0.4mm}{+}及OLI传感器,下面是一个关于TM传感器的示例:
%	\begin{table}
%		\small
%		\newcommand{\tabincell}[2]{\begin{tabular}{@{}#1@{}}#2\end{tabular}}
%		\caption{Extraction Result of Indices with TM Sensor}  
%		\centering
%		\begin{tabular}{ccccc}
%			\toprule
%			Index	& Best Threshold &\tabincell{c}{Area of Honghu\\Lake in 1998/$km^2$}&Overall Accuracy&Kappa \\ \midrule
%			Reference	&--	&294.294&	100.000\%&	1.000\\
%			NDWI&	-0.341&	294.306&	97.437\%&	0.949\\
%			MNDWI&	\raisebox{0.2mm}{+}0.021&	294.277&	97.528\%&	0.951\\
%			AWEI&	\raisebox{0.2mm}{+}0.000&	281.473&	96.327\%&	0.927\\
%			NDVI&	\raisebox{0.2mm}{+}0.355&	294.311&	97.278\%&	0.946\\
%			WRI&	\raisebox{0.2mm}{+}0.662&	294.303&	97.489\%&	0.950\\
%			NDMI&	\raisebox{0.2mm}{+}0.155&	294.252&	94.925\%&	0.899\\ \bottomrule
%		\end{tabular}
%	\end{table}
%\end{frame}

\begin{frame}[fragile]
	\frametitle{精度评价}
	\framesubtitle{水体提取-直方图分割}
	产生1000个随机点,每个区域500个。
	\begin{table}
		\centering
		\newcommand{\tabincell}[2]{\begin{tabular}{@{}#1@{}}#2\end{tabular}}
		\caption{Accuracy Result of Automatic Water Extraction(OLI Sensor)}
		\begin{tabular}{ccccc}
			\toprule
			Research Area 	&\tabincell{c}{Overall\\Accuracy} &\tabincell{c}{  Producer's\\Accuracy} &\tabincell{c}{User's\\Accuracy}& Kappa \\ \midrule
			Liangzihu Lake	&82.200\%   &   72.766\%     &  87.245\%  & 0.639\\
			Futouhu Lake	&95.400\%	&83.969\%	&98.214\%	&0.875\\
			Integrate		&88.800\%	&76.776\%	&91.234\%	&0.750\\ \bottomrule
		\end{tabular}
		\label{tiqu_jindu}
	\end{table}
\end{frame}

\begin{frame}[fragile]
	\frametitle{精度评价}
	\framesubtitle{SVM分类}
	评定了TM和OLI传感器的分类精度。\\
	下面是两个例子:
	\begin{figure}
		\small
		\setcounter{subfigure}{0}
		\centering
		\subfigure[Spec]{
			\includegraphics[width=0.15\textwidth]{./figure/lzh_oli/spec.pdf}
		}
		\subfigure[P+A]{
			\includegraphics[width=0.15\textwidth]{./figure/lzh_oli/PA.pdf}
		}
		\subfigure[Spec+P+A]{
			\includegraphics[width=0.15\textwidth]{./figure/lzh_oli/spec_PA.pdf}
		}
		\subfigure[P+A+ger]{
			\includegraphics[width=0.15\textwidth]{./figure/lzh_oli/PA_ger.pdf}
		}
		\subfigure[P+A+rec]{
			\includegraphics[width=0.15\textwidth]{./figure/lzh_oli/PA_rec.pdf}
		}
		\subfigure[P+A+cur]{
			\includegraphics[width=0.15\textwidth]{./figure/lzh_oli/PA_cur.pdf}
		}
		\caption{Classification Result of West Liangzihu Lake Area}
	\end{figure}
\end{frame}

\begin{frame}[fragile]
		\frametitle{精度评价}
	\framesubtitle{SVM分类}
	\begin{figure}
		\small
		\centering
		\subfigure[Spec]{
			\includegraphics[width=0.15\textwidth]{./figure/fth_oli/spec.pdf}
		}
		\subfigure[P+A]{
			\includegraphics[width=0.15\textwidth]{./figure/fth_oli/PA.pdf}
		}
		\subfigure[Spec+P+A]{
			\includegraphics[width=0.15\textwidth]{./figure/fth_oli/spec_PA.pdf}
		}
\end{figure}
\begin{figure}
	\small
	\centering
		\subfigure[P+A+ger]{
			\includegraphics[width=0.15\textwidth]{./figure/fth_oli/PA_ger.pdf}
		}
		\subfigure[P+A+rec]{
			\includegraphics[width=0.15\textwidth]{./figure/fth_oli/PA_rec.pdf}
		}
		\subfigure[P+A+cur]{
			\includegraphics[width=0.15\textwidth]{./figure/fth_oli/PA_cur.pdf}
		}
	\caption{Classification Result of South Futouhu Lake Area}
	\end{figure}
\end{frame}

\begin{frame}[fragile]
		\frametitle{精度评价}
	\framesubtitle{SVM分类}
	\begin{table}
		\footnotesize
		\newcommand{\tabincell}[2]{\begin{tabular}{@{}#1@{}}#2\end{tabular}}  
		\centering
		\caption{Classification Accuracy of West Liangzihu Lake Area with OLI Sensor}
		\begin{tabular}{ccccccc}
			\toprule
			&  &\multicolumn{2}{c}{\tabincell{c}{Producer's\\ Accuracy}}&\multicolumn{2}{c}{\tabincell{c}{User's\\Accuracy}} &\\ \midrule 
			\tabincell{c}{Feature\\Combination}&\tabincell{c}{Overall\\Accuracy}&\tabincell{c}{Lake\\Waterbody}&Aquaculture &\tabincell{c}{Lake\\Waterbody}&Aquaculture&Kappa\\
			spec&	91.818\%&	90.909\%&	100.000\%&	100.000\%&	55.000\%&	0.667\\
			P+A&	90.000\%&	88.889\%&	100.000\%&	100.000\%&	50.000\%&	0.615\\
			P+A+spec&	93.636\%&	96.970\%&	63.636\%&	96.000\%&	70.000\%&	0.632\\
			P+A+ger&	97.273\%&	97.980\%&	90.909\%&	98.980\%&	83.333\%&	0.854\\
			P+A+rec&	92.727\%&	91.919\%&	100.000\%&	100.000\%&	57.895\%&	0.695\\
			P+A+cur&	97.273\%&	97.980\%&	90.909\%&	98.980\%&	83.333\%&0.854\\   \bottomrule
		\end{tabular}
	\end{table}
\end{frame}

\begin{frame}[fragile]
	\frametitle{精度评价}
	\framesubtitle{SVM分类}
	\begin{table}
		\footnotesize
		\newcommand{\tabincell}[2]{\begin{tabular}{@{}#1@{}}#2\end{tabular}}  
		\centering
		\caption{Classification Accuracy of South Futouhu Lake Area with OLI Sensor}
		\begin{tabular}{ccccccc}
			\toprule
			&  &\multicolumn{2}{c}{\tabincell{c}{Producer's\\ Accuracy}}&\multicolumn{2}{c}{\tabincell{c}{User's\\Accuracy}} &\\ \midrule 
			\tabincell{c}{Feature\\Combination}&\tabincell{c}{Overall\\Accuracy}&\tabincell{c}{Lake\\Waterbody}&Aquaculture &\tabincell{c}{Lake\\Waterbody}&Aquaculture&Kappa\\
			spec&	36.842\%&	0.000\%&	100.000\%&	0.000\%&	36.842\%&	0.000\\
			P+A&	81.871\%&	73.148\%&	96.825\%&	97.531\%&	67.778\%&	0.642\\
			P+A+spec&	88.889\%&	88.889\%&	88.889\%&	93.204\%&	82.353\%&	0.765\\
			P+A+ger&	71.345\%&	91.667\%&	36.508\%&	71.223\%&	71.875\%&	0.314\\
			P+A+rec&	90.643\%&	87.963\%&	95.238\%&	96.939\%&	82.192\%&	0.805\\
			P+A+cur&	85.380\%&	86.111\%&	84.127\%&	90.291\%&	77.941\%&	0.691\\
			\bottomrule
		\end{tabular}
	\end{table}
\end{frame}

%\section{讨论}
%\subsection{讨论}
%
%\begin{frame}
%\frametitle{讨论}
%\begin{columns}
%	\begin{column}{0.4\textwidth}
%		\begin{figure}
%		%\begin{minipage}[t]{0.2\textwidth}
%		\includegraphics[width=0.5\textwidth]{./figure/classify_result/error_origin.pdf}\\
%		\includegraphics[width=0.5\textwidth]{./figure/classify_result/error_classify.pdf}
%		%\end{minipage}
%		\caption{Discussion Area}
%		\end{figure}
%	\end{column}
%	\begin{column}{0.6\textwidth}
%		\begin{block}{分析}
%		\scriptsize
%		\begin{itemize}
%		\item 植被影响:对象边界复杂度$\uparrow$,被误分类为水体。(北部方形对象)
%		
%		\textbf{本质:混合像元}\\
%		\textbf{解决方案:像元分解}
%		\item 分辨率较低:提取不完全,对象边界复杂度$\downarrow$,被误分类为鱼塘。(南部河流)
%		
%		\textbf{解决方案:高分辨率影像}
%		\item 分辨率较低:对象粘连,对象边界复杂度$\uparrow$,被误分类为水体。(西部鱼塘聚集区)
%		
%		\textbf{解决方案:高分辨率影像}
%		\item 分辨率较低 \& 提取误差:自然水体也会存在规则边界,对象边界复杂度$\downarrow$, 被误分类为鱼塘。(东部自然水体)
%		
%		\textbf{解决方案:}
%		\begin{itemize}
%		\scriptsize
%		\item [-] 更高分辨率影像。
%		\item [-] \alert{面向场景},在语义层面判断。
%		\item [-] \alert{深度学习(Deep Learning, DL)},自动获取特征。
%		\end{itemize}
%	
%		\end{itemize}
%		\end{block}
%	\end{column}
%\end{columns}
%\end{frame}

\section{尾声}
\subsection{尾声}
\begin{frame}
	\frametitle{总结}
	\small
	\begin{itemize}
%	\item [\uppercase\expandafter{\romannumeral1}]本文根据面向对象的思想,从光谱和几何两个方面获取自动化提取的水体对象的特征,通过将不同尺度、不同维度的特征进行组合并利用SVM分类算法,获取了不同特征组合的分类结果。
	
	\item [\uppercase\expandafter{\romannumeral1}]自动化水体提取效果较好,因为确定阈值时考虑了整个图像上像元灰度值的分布情况,并且选择了使得像元类别差异最大的阈值,便于之后的分类。
	
	\item [\uppercase\expandafter{\romannumeral2}]不同的特征组合在各自较优的模型参数下进行分类会得到不同结果,表明数据点在不同特征空间中有着不同的分布。 
	
	\item [\uppercase\expandafter{\romannumeral3}]纯光谱特征对区分水体对象几乎不起作用,面积周长作为基本几何特征,可以对水体对象初步分割;而紧凑性,矩形度,弯曲程度这些高级几何特征则是在几何层面的更高维度对对象进行探测。
	
	\item [\uppercase\expandafter{\romannumeral4}]从对象弯曲性这一角度来衡量水体对象之间的差异较其他特征更为有效,其受由于分辨率因素导致的对象粘连引起的对象形态改变的影响较弱,且在30m分辨率影像条件下的不同场景均有较强的的适应性。
	
	\item [\uppercase\expandafter{\romannumeral5}]可以考虑通过对象的弯曲程度这种微观特征从几何角度来衡量自然水体与人工水体的差异,并有望应用到其他类似的自然地物与人工地物类型的分类上。
	
	\item [\uppercase\expandafter{\romannumeral6}]\alert{不足}:训练集中水体对象较少,数据欠丰富;分辨率有限、植被覆盖等,影响提取效果。
	\end{itemize}
\end{frame}
\begin{frame}
	\centering
	\Huge
	烦请各位不吝赐教\\[5ex]
	感谢这四年,谢谢你们!
\end{frame}
\end{document}